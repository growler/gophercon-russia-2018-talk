% a poor man's sequence diagram
\usetikzlibrary{arrows,arrows.meta,calc,positioning,backgrounds}
\tikzset{>={Stealth[length=1.2ex, width=.6ex]}}

\makeatletter
\newenvironment{seqdiag}[2][]{%
\definecolor{diagramcolor}{rgb}{0.62,0.00,0.184}%
\pgfkeys{/sd/.cd,%
  height/.initial=9,%
  #1}%
% \msg{Position}{From}{To}{Comment}
\newcommand{\msg}[4]{%
  \def\@srcw{\csname line@width@##2\endcsname}%
  \def\@dstw{\csname line@width@##3\endcsname}%
  \draw let \p1=(line@origin@##2),%
            \p2=(line@origin@##3),%
            \p3=($(curpos)+(0,##1)$),
            \n1={(\x1 < \x2) ? (\x1 + \@srcw) : (\x1 - \@srcw)},%
            \n2={(\x1 < \x2) ? (\x2 - \@dstw) : (\x2 + \@dstw)}%
        in%
            [->,draw=diagramcolor, line cap=rect, line width=0.5pt]%
            (\n1,\y3) -- node[midway, fill=white] {##4} (\n2,\y3)%
            coordinate (curpos) at (\p3);%
}%
\newcommand{\proc}[4]{%
  \draw let
        \p1=($(line@origin@##2)-(2ex,0)$),
        \p2=($(line@origin@##3)+(2ex,0)$),
        \p3=($(curpos) + (0,##1)$)
   in
        [draw, line width=0.5pt, double distance=0.5ex]
        (\x1,\y3) -- node[midway, fill=white] {##4} (\x2,\y3)
        coordinate (curpos) at (\p3);%
}%
\newcommand{\enclose}[3]{
  \begin{scope}[on background layer]
    \path let \p1=(current bounding box.north), \p2=(current bounding box.south) in
          [draw, dotted, thin] (##1,\y1) rectangle (##2,\y2)
          (##1, \y1) -- node[below] {##3} (##2,\y1);
  \end{scope}
}
% \instance{position}{Name}{Comment}
\newcommand{\instance}[3]{%
  \begin{scope}[yscale = 1, local bounding box=inst@##2]%
    \node[draw] at (##1cm, 0) {##2};%
  \end{scope}%
  \path let \p1=(inst@##2.south) in coordinate (line@origin@##2) at (\x1, 2ex);%
}
% \actor{position}{name}{comment}
\newcommand{\actor}[3]{%
  \begin{scope}[yscale = 1, inner sep=0, local bounding box=inst@##2]%
    \node (name) at (##1cm, 0) {##2};%
    \node[above=3pt of name] {##3};%
  \end{scope}%
  \path let \p1=(inst@##2.south) in coordinate (line@origin@##2) at (\x1, 2ex);%
}%
\newcommand{\eventline}[1]{
  \expandafter\def\csname line@width@##1\endcsname{0.25pt}%
  \pgfkeysgetvalue{/sd/height}{\@sd@height}%
  \path let \p1=(line@origin@##1)
        in
            coordinate (origin) at (\p1)
            coordinate (finish) at ($(\p1) + (0,\@sd@height)$);%
  \draw[draw=diagramcolor, line width=0.5pt, line cap=rect, dashed] (origin) -- (finish);%
}%
\newcommand{\eventthread}[3]{
  \expandafter\def\csname line@width@##1\endcsname{2.25pt}%
  \pgfkeysgetvalue{/sd/height}{\@sd@height}%
  \path let \p1=(line@origin@##1)
        in
            coordinate (origin) at (\p1)
            coordinate (start) at ($(\p1) + (0,##2cm)$)
            coordinate (stop) at ($(\p1) + (0,##3cm)$)
            coordinate (finish) at ($(\p1) + (0,\@sd@height)$)
            coordinate (left) at ($(start) - (2pt,0)$)
            coordinate (right) at ($(stop) + (2pt,0)$);
  \draw[diagramcolor, line width=0.5pt, line cap=rect, dashed] (origin) -- (start);%
  \draw[diagramcolor, line width=0.5pt] (left) rectangle (right);
  \draw[diagramcolor, line width=0.5pt, line cap=rect, dashed] (stop) -- (finish);%
}%
\begin{tikzpicture}[yscale=-1]%
  \path coordinate (curpos) at (0,2ex);
  #2%
}{%
\end{tikzpicture}%
}
\makeatother
%%% Local Variables:
%%% mode: latex
%%% TeX-master: t
%%% End:
